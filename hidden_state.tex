\documentclass{article}
\pagestyle{empty}
\begin{document}

$n_{h,k}$: number of founder bases of type h called as type k

N: = $\sum_{h,k}{n_{h,k}}$

o: observed configuration (ordered array of N symbols)

F: founder base count (vector of 4 integers)

O: observed base count (derived from o, vector of 4 integers)

L: path length through simplex from O to F

e: probability of miscall

\begin{eqnarray}
Pr(o, L, F) & \equiv & MC(\{n_{h,k}\}) Path(L, N, q) Prior(F|\alpha) \\[6ex]
MC(\{n_{h,k}\}) & \equiv & \prod_h \frac{n_{h,\cdot}!}{\prod_k n_{h,k}!} \\[6ex]
Path(L, N, e) & \equiv & e^L (1-e)^{N-L} \\[6ex]
Prior(F|\alpha) & \equiv & \int_p{Pr(x|p)Pr(p|\alpha)dp} \nonumber \\
& = & \frac{N!}{\prod_h n_h!} 
\frac{\Gamma(A)}{\Gamma(N+A)}
\prod_h \frac{\Gamma(n_h+\alpha_h)}{\Gamma(\alpha_h)} \\[6ex]
Pr(o,F) & \equiv & \sum_L Pr(o,L,F)
\end{eqnarray}

Some derivations in 2D.  Let (x,y) represent hidden counts ${n_0,n_1}$
and let $\check{x} \equiv x-1$ and $\hat{y} \equiv y+1$ in the
following.

\begin{eqnarray}
  \frac{Path(N,L,e)}{Path(N,L+2,e)}
  & = & \frac{e^L(1-e)^{N-L}}{e^{L+2}(1-e)^{N-L-2}} \nonumber \\
  & = & \frac{(1-e)^2}{e^2} \\[4ex]
  \frac{Prior(\{\check{x},\hat{y}\})}{Prior(\{x,y\})}
  & = &
    \frac{\frac{N!}{\check{x}!\hat{y}!}}{\frac{N!}{x!y!}}
    \frac{\frac{\Gamma(A)}{\Gamma(N+A)}}{\frac{\Gamma(A)}{\Gamma(N+A)}}
    \frac{
      \frac{\Gamma(\check{x}+\alpha_0)}{\Gamma(\alpha_0)}
    }{
      \frac{\Gamma(x+\alpha_0)}{\Gamma(\alpha_0)}
    }
    \frac{
      \frac{\Gamma(\hat{y}+\alpha_1)}{\Gamma(\alpha_1)}
    }{
      \frac{\Gamma(y+\alpha_1)}{\Gamma(\alpha_1)}
    } \nonumber \\[2ex]
    & = &
    \frac{x!}{\check{x}!}
    \frac{y!}{\hat{y}!}
    \frac{\hat{y}+\alpha_1}{x+\alpha_0} \nonumber \\
    & = &
    \frac{x}{x+\alpha_0}
    \frac{\hat{y}+\alpha_1}{\hat{y}}
\end{eqnarray}

Consider now a heuristic way to evaluate the posterior $Pr(o,F) \equiv
\sum_L Pr(o,L,F)$. In 2D, the non-zero components exist for every
other value of L, i.e. (0,2,4,...) or (1,3,5,...). The individual
components $Pr(o,L,F)$ differ from each other in the factors of Path
probability and $MC()$, the multinomial coefficient, which is the number
of configurations $h$ that are compatible with the path
length.

For illustration, let's assume the configuration $o$ has 10A and 90C.
Define hA as the number of 'A' in the group of hidden configurations.
Define v as the number of 'A' elements in $h$ that overlap 'A'
elements in o.  It follows that $L \equiv (10-v) + (hA-v) = hA + 10 -
2v$, and $v = \frac{hA + 10 - L}{2}$.  And, $MC(hA,L) \equiv {10
  \choose v}{90 \choose (hA-v)}$.

Increasing L by 2 changes (decreases) the Path probability by factor
$\frac{e^2}{(1-e)^2}$.

%% \begin{eqnarray}
%% \frac{10 \choose (v-1)}{10 \choose v} \frac{90 \choose (hA-v+1)}{90 \choose (hA-v)}
%% & = &
%% \frac{v}{9-v} \frac{90-

\end{document}
